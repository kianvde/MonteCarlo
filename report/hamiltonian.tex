Our goal is to find an upper bound for the energy of the ground state of the hydrogen molecule. To do this, we use the variational approach, which states that 
\begin{align}
E = \frac{{\left\langle \psi_{tr}  \right|\hat H\left| \psi_{tr}  \right\rangle }}{{\left\langle \psi_{tr}  \right|\left. \psi_{tr}  \right\rangle }} = \frac{{\int {d{{\vec r}_1}\int {d{{\vec r}_2}} {\psi_{tr} ^ * }\hat H\psi_{tr} } }}{{\left\langle \psi_{tr}  \right|\left. \psi_{tr}  \right\rangle }} \ge {E_g}.
\label{eq:main}
\end{align}
Our goal is now to evaluate this integral with the help of the Monte Carlo method. 
\subsection{Variational (Quantum) Monte Carlo}
Monte Carlo integration is based on the random sampling of the integrands evaluation points. For higher dimensional integrals this method converges faster than standard numerical methods as the distribution uniformity of the random points is higher than for example a cubical grid. To improve on this integration scheme the integral should be rewritten in a form where we can sample over a distribution close to the shape of the integrand, this ensures that the regions that contribute strongly to the integral are evaluated more exactly. This method is called importance sampling Monte Carlo and will be used in our evaluation of the hydrogen molecule’s energy. \\
In our case we would like to sample over  $|{\phi _{tr}}{|^2}$ as it will be close to the integrand in the energy expression $\phi _{_{tr}}^*H{\phi _{tr}}$. To achieve this the Metropolis-Hastings algorithm will be used.\\
\subsubsection{Metropolis-Hastings}
To sample from the probability distribution (${p \propto |{\psi _{tr}}{|^2}}$) needed to calculate the integral we use the Metropolis-Hastings algorithm, which is a Markov chain Monte Carlo based method~\cite{robert2013monte}. The main advantage of this method is that we do not have to normalize the wavefunction as this algorithm draws from a valid (normalized) probability distribution proportional to the function supplied (in our case ${|{\psi _{tr}}{|^2}}$). The general algorithm as implemented in our code can be stated as:
\begin{enumerate}
\item randomly initiate the values for ${\overrightarrow {{r_1}} }$ and ${\overrightarrow {{r_2}} }$ 
\item  propose new ${\overrightarrow {{r_1}} }$ and ${\overrightarrow {{r_2}} }$ by stepping both vectors in a random direction where the step size is drawn from a normal distribution
\item calculate the acceptance ratio: $|{\psi _{tr}}|_{new}^2/|{\psi _{tr}}{|^2}$
\item if the ratio is bigger than one accept the new coordinates, if the ratio is smaller than one accept with probability given by the ratio
\end{enumerate}
For the convergence of this algorithm we will have to make smart decisions for the step size used in  step two. It is important to not have the step size too small as then the coordinate space will be traced out slowly. The same thing will happen if the step size is too big because then the acceptance rate will be low. In our code the standard deviation for the step size is updated according to
\begin{align}
\sigma  = \sigma  + \delta  (A - 0.5),
\end{align}
A is the average acceptance rate and $\delta$ is a factor between zero and one, in our case we have used $\delta=0.05$. This will push the acceptance rate towards 50\% ensuring fast convergence of the algorithm.



\subsection{Hamiltonian of the hydrogen molecule}
Before we use this method, however, we have to rewrite Equation~\ref{eq:main} somewhat. We start by writing down the Hamiltonian for the hydrogen molecule. We assume that the protons do not move (Born-Oppenheimer approximation). Then the Hamiltonian consists of two single-electron Hamiltonians and two extra terms to account for electron-electron repulsion and proton-proton repulsion respectively. This gives the following result:
\begin{align}	   
\hat H =& {\hat H_1} + {\hat H_2} + {\hat H_{ee}} + {\hat H_{pp}}\\
\hat H =& \left( { - \frac{{{\hbar ^2}}}{{2m}}\nabla _1^2 - \frac{{{e^2}}}{{4\pi {\varepsilon _0}}}\frac{1}{{{r_{1L}}}} - \frac{{{e^2}}}{{4\pi {\varepsilon _0}}}\frac{1}{{{r_{1R}}}}} \right) + \left( { - \frac{{{\hbar ^2}}}{{2m}}\nabla _2^2 - \frac{{{e^2}}}{{4\pi {\varepsilon _0}}}\frac{1}{{{r_{2L}}}} - \frac{{{e^2}}}{{4\pi {\varepsilon _0}}}\frac{1}{{{r_{2R}}}}} \right)\nonumber\\
& + \left( {\frac{{{e^2}}}{{4\pi {\varepsilon _0}}}\frac{1}{{{r_{12}}}}} \right) + \left( {\frac{{{e^2}}}{{4\pi {\varepsilon _0}}}\frac{1}{{{r_{LR}}}}} \right).
\end{align}\\
In this equation we have defined ${r_{1L}} = \left| {{{\vec r}_1} - {{\vec r}_L}} \right|$, etc., where $r_L$ is the position vector of the left-most proton. To simplify the units somewhat, we introduce atomic units, where lengths are given in terms of the Bohr radius and energies are given in terms of twice the ionization energy of hydrogen:
\begin{align}
{a_0} =& \frac{{4\pi {\varepsilon _0}{\hbar ^2}}}{{m{e^2}}} = 0.529\mathring{A}\\
E =& 2 \cdot \frac{{{e^2}}}{{4\pi {\varepsilon _0}{a_0}}} = 2 \cdot \left( { - 13.6eV} \right) =  - 27.2eV
\end{align}\\
In these units, we can write the Hamiltonian as follows \cite{MSU_paper}:
\begin{align}
\hat H =  - \frac{1}{2}\left( {\nabla _1^2 + \nabla _2^2} \right) - \frac{1}{{{r_{1L}}}} - \frac{1}{{{r_{1R}}}} - \frac{1}{{{r_{2L}}}} - \frac{1}{{{r_{2R}}}} + \frac{1}{{{r_{12}}}} + \frac{1}{{{r_{LR}}}}
\end{align}
Now, we can choose our axes in such a way that the two protons are located symmetrically along the x-axis at $-s/2$ and $+s/2$, with $s$ the separation between the protons. In that case, we can write the distances between an electron and a proton as
\begin{align}
{r_{1L}} =& \left| {{{\vec r}_1} + \frac{s}{2}\hat x} \right|\\{r_{1R}} =& \left| {{{\vec r}_1} - \frac{s}{2}\hat x} \right|
\end{align}
Now that we have the Hamiltonian, we can now try to write down a trial wave function to use for variational calculus. Luckily, this has already been done, so we can immediately write down the final result\cite{MSU_paper}:
\begin{align}
\Psi \left( {{{\vec r}_1},{{\vec r}_2}} \right) = {\phi _1}\left( {{{\vec r}_1}} \right){\phi _2}\left( {{{\vec r}_2}} \right)\psi_{tr} \left( {{{\vec r}_1},{{\vec r}_2}} \right)
\end{align}
The trial wave function consists of two single-electron terms (one that only depends on $r_1$, and one that only depends on $r_2$) and an interaction term $\psi_{tr}$. The single-electron wave functions are given by
\begin{align}
{\phi _1}\left( {{{\vec r}_1}} \right) =& {e^{ - {{{r_{1L}}} \mathord{\left/
 {\vphantom {{{r_{1L}}} a}} \right.
 \kern-\nulldelimiterspace} a}}} + {e^{ - {{{r_{1R}}} \mathord{\left/
 {\vphantom {{{r_{1R}}} a}} \right.
 \kern-\nulldelimiterspace} a}}} = {\phi _{1L}} + {\phi _{1R}},\\
 {\phi _2}\left( {{{\vec r}_2}} \right) =& {e^{ - {{{r_{2L}}} \mathord{\left/
 {\vphantom {{{r_{2L}}} a}} \right.
 \kern-\nulldelimiterspace} a}}} + {e^{ - {{{r_{2R}}} \mathord{\left/
 {\vphantom {{{r_{2R}}} a}} \right.
 \kern-\nulldelimiterspace} a}}} = {\phi _{2L}} + {\phi _{2R}}.
\end{align}
These wave functions correspond to two non-interacting electrons orbiting two protons. However, this does not accurately describe our system. We must also add an interaction term
\begin{align}
\psi_{tr} \left( {{{\vec r}_1},{{\vec r}_2}} \right) = {e^{\frac{{\left| {{{\vec r}_1} - {{\vec r}_2}} \right|}}{{\alpha \left( {1 + \beta \left| {{{\vec r}_1} - {{\vec r}_2}} \right|} \right)}}}}
\end{align}
In this term, $\alpha$ and $\beta$ are parameters that we will optimize to find the minimum of the energy through the variational approach.\\
\newpage
\subsection{Local energy}
Now we re-examine the integral that we want to evaluate (Equation \ref{eq:main}) in the variational approach, which is
\begin{align}
E = \frac{{\int {d{{\vec r}_1}\int {d{{\vec r}_2}} {\psi_{tr} ^ * }\hat H\psi_{tr} } }}{{\left\langle \psi_{tr}  \right|\left. \psi_{tr}  \right\rangle }} = \int {d{{\vec r}_1}\int {d{{\vec r}_2}} \frac{{{\psi_{tr} ^ * }\psi_{tr} }}{{\left\langle \psi_{tr}  \right|\left. \psi_{tr}  \right\rangle }} \cdot \frac{{\hat H\psi_{tr} }}{\psi_{tr} }}
\label{eq:var_int}.
\end{align}
We define the local energy $\epsilon$ and the weight $\omega$ as
\begin{align}
\varepsilon \left( {{{\vec r}_1},{{\vec r}_2},s} \right) =& \frac{{\hat H\psi_{tr} }}{\psi_{tr} },\\
\omega \left( {{{\vec r}_1},{{\vec r}_2},s} \right) =& \frac{{{{\left| \psi_{tr}  \right|}^2}}}{{\left\langle \psi_{tr}  \right|\left. \psi_{tr}  \right\rangle }}.
\end{align}
Then we can write the integral of Equation \ref{eq:var_int} as 
\begin{align}
E = \int {d{{\vec r}_1}\int {d{{\vec r}_2}} \omega \left( {{{\vec r}_1},{{\vec r}_2},s} \right)\varepsilon \left( {{{\vec r}_1},{{\vec r}_2},s} \right)}.
\end{align}
Our final step is to write down the local energy of the system. The derivation of the local energy is a bit of work (see \cite{MSU_paper} or appendix A), but eventually one can find as final result
\begin{align}
\varepsilon  =&  - \frac{1}{{{a^2}}} + \frac{1}{{a{\phi _1}}}\left[ {\frac{{{\phi _{1L}}}}{{{r_{1L}}}} + \frac{{{\phi _{1R}}}}{{{r_{1R}}}}} \right] + \frac{1}{{a{\phi _2}}}\left[ {\frac{{{\phi _{2L}}}}{{{r_{2L}}}} + \frac{{{\phi _{2R}}}}{{{r_{2R}}}}} \right] + \left( {\frac{{{\phi _{1L}}{{\hat r}_{1L}} + {\phi _{1R}}{{\hat r}_{1R}}}}{{{\phi _1}}} - \frac{{{\phi _{2L}}{{\hat r}_{2L}} + {\phi _{2R}}{{\hat r}_{2R}}}}{{{\phi _2}}}} \right)\left( {\frac{{{{\hat r}_{12}}}}{{2a{\gamma ^2}}}} \right)\nonumber\\
 &- \frac{{\left( {1 + 4\beta } \right){r_{12}} + 4}}{{4{\gamma ^4}{r_{12}}}} - \frac{1}{{{r_{1L}}}} - \frac{1}{{{r_{1R}}}} - \frac{1}{{{r_{2L}}}} - \frac{1}{{{r_{2R}}}} + \frac{1}{{{r_{12}}}} + \frac{1}{{{r_{LR}}}}.
\end{align}
Now, it is important to note that this result is not always bounded. Since we do not want a solution that blows up, we can derive a set of conditions to keep the solution bounded. While deriving the local energy, we have already found one condition for the case $r_{12}\rightarrow0$, namely $\alpha=0$.
We also have a condition for the case $r_{1L}\rightarrow0$:
\begin{align}
\mathop {\lim }\limits_{{r_{1L}} \to 0} \left| \varepsilon  \right| < \infty  &\Rightarrow \frac{1}{{a{\phi _1}}}\left[ {\frac{{{\phi _{1L}}}}{{{r_{1L}}}}} \right] - \frac{1}{{{r_{1L}}}} = 0\nonumber\\& \Rightarrow \frac{{{e^{ - {r_{1L}}/a}}}}{{a\left( {{e^{ - {r_{1L}}/a}} + {e^{ - {r_{1R}}/a}}} \right)}} = 1\nonumber\\&
 \Rightarrow a = \frac{1}{{1 + {e^{ - \left| {{{\vec r}_L} - {{\vec r}_R}} \right|/a}}}}.
\end{align}
The other possibilities ($r_{1R}\rightarrow0$, etc) give the same condition, which is called the Coulomb cusp condition. With this condition, we can now eliminate $a$ as a variable from the local energy by solving the above equation.
However, this is easier said than done. The non-linear form of this equation means that it is not possible to solve for a analytically. Therefore, we use Newton-Rhapson to numerically find the value for a for which the above equation holds.
\noindent The iteration scheme for Newton-Rhapson is given by
\begin{align}
{a^{n + 1}} = {a^n} - \frac{{f\left( {{a^n}} \right)}}{{f'\left( {{a^n}} \right)}},
\end{align}
where $f$ is given by
\begin{align}
f\left( a \right) = \frac{1}{{1 + {e^{ - s/a}}}} - a.
\end{align}
We also need the derivative of f, which is
\begin{align}
f'\left( a \right) =  - \frac{{s{e^{ - s/a}}}}{{{{\left( {a + a{e^{ - s/a}}} \right)}^2}}} - 1.
\end{align}
Now we can use the above scheme to find the value of $a$ for a given $s$. Therefore, since we want to know the minimum of the local energy for every $s$, we only need to find one parameter, namely $\beta$. This parameter will now be found using the Monte Carlo method. 
